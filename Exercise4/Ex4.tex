\documentclass[a4paper,11pt]{report}

%\usepackage[options]{package}
\usepackage{graphicx}
\usepackage{color} 
\usepackage[dvipsnames]{xcolor}
\colorlet{purple}{purple}

\begin{document}


\section{\color{olive}Exercise 4: Behaviour Analysis of Circuits Including a 74HC02 Gate}

The 74HC02 is an integrated circuit of NOR gates. Firstly, the propagation delays and the transition times are measured for this gate in a no-load output condition. Then these parameters are measured in the case in which the gate is connected in the circuit in Figure. %HACER REFERENCIA A LA FOTITO
In the following table, the results of the mentioned measures are shown.
 %PONER FOTITO DE LA NOR (no load)
%foto circuito
%foto circuito con capacitor

\subsubsection{\color{red}Propagation and Transition Times' Measurements}

\begin{tabular}{|c|c|c|}
%\label{first meas}
\hline
 &No-load & Loaded with circuit \\ %aclarar haciendo referencia a la foto del circuito
\hline
\hline
$t_{pHL}$ & 1,5ns & 1,05ns \\
\hline
$t_{pLH}$ & 12,5ns & 14,4ns\\
\hline
$t_{f}$ & 32ns & 30,7ns\\
\hline
$t_{r}$ & 41,4ns & 43,2ns \\
\hline
\end{tabular}

In the previous table it can be seen that the four parameters remain practically the same while being the 74HC02 in the no-load situation and incorporated in the circuit. However, the small differences have opposite behaviours depending on the input signal's edge. In the case of a rising edge in the input signal, the propagation delay $t_{pLH}$ and the transition time $t_{r}$ are bigger if the 74HC02 forms part of the circuit, than if the 74HC02 is in a no-load situation.

\subsubsection{\color{red}Circuit's Response to Frequency Increment}



\subsubsection{\color{red}Alimentation Voltage and Temperature of the IC}


Se pone un capacitor de desacople para que cuando se pida corriente, la fuente VCC no sea alterada creando ripples no deaseados 


\end{document} 