%AGREGAR:
%fotos, graficos
%explicar y mostrar simulaciones
%comparar para el noise margin la simulacion con las distintas R
%lo del capacitor de 1 nF

\documentclass[a4paper,11pt]{report}

%\usepackage[options]{package}
\usepackage{graphicx}
\usepackage{color} 
\usepackage[dvipsnames]{xcolor}
\colorlet{purple}{purple}

\begin{document}

\section{\color{olive}Exercise 1: Design and Implementation of NOT Gates Using Transistors} %Despues borrar el Exercise 1 del titulo y que queden solo los titulos de los ejs en color olive.


\subsubsection{\color{red}High-Level and Low-Level Input Voltages}
The high-level input voltage ($V_{IH}$) is the minimum input voltage that is considered as high, while the low-level input voltage ($V_{IL}$)  is the maximum input voltage that is considered as low.

\subsubsection{\color{red}High-Level and Low-Level Output Voltages}
The high-level output voltage ($V_{OH}$) is the minimum output voltage that the circuit provides as a high, while the low-level output voltage ($V_{OL}$) is the maximum output voltage that the circuit provides as a low.

\subsubsection{\color{red}Noise Margin}
The high noise margin ($NM_{H}$) is the gap between the high-level input voltage and the high-level output voltage, while the low noise margin ($NM_{L}$) is the gap between the low-level output voltage and the low-level input voltage.
$$NM_{H} = V_{OH} - V_{IH}$$
$$NM_{L} = V_{IL} - V_{OL}$$

\subsubsection{\color{red}Propagation Delays}
For this assignment's meassures, when the input changes from low to hight and the output from high to low, the high-to-low propagation delay is considered as the time between the moment in which the input voltage reaches the $90\% $ of its maximum high value, until the moment in which the output voltage reaches the $10\%$ of its maximum high value.
$$t_{pHL} = t_{10\%V_{maxO}} - t_{90\%V_{maxI}}$$
In the case in which the input goes from high to low and the output from low to high, the low-to-high propagation delay is considered as the time between the moment in which the input voltage reaches the $10\%$ of its maximum high value, until the moment in which the output voltage reaches the $90\%$ of its maximum high value.
$$t_{pLH} = t_{90\%V_{maxO}} - t_{10\%V_{maxI}}$$

\subsubsection{\color{red}Transition Times}
The high-to-low transition time or fall time ($t_{f}$) is the time that it takes the output voltage to go from its high maximum value to its low minimum value, while the low-to-high transition time or rise time ($t_{r}$) is the time that it takes for it to change from its low minimum value to tis high maximum value.


\subsubsection{\color{red}Maximum Output Current}
%ESCRIBIR ACA LO DE LA DERIVADA DE LA TENSION EN EL OSCILOSCOPIO

\subsection{\color{purple}Measurements}


\subsection{\color{purple}Using a BJT NPN 337 Transistor}

\subsubsection{\color{red}Without Load Connected to the Output}

\subsubsection{\color{red}With a $1 nF$ Capacitor Connected to the Output}


\subsection{\color{purple}Using a BJT PNP 327 Transistor}

\subsubsection{\color{red}Without Load Connected to the Output}


\begin{tabular}{|c|c|c|c|c|}
\hline
 & NPN & NPN with capacitor &PNP& PNP with capacitor\\
\hline
\hline
$V_{IH} (V)$ & 0,9 & 0,9 & 4,5 & 4,6  \\
\hline
$V_{IL} (V)$ & 0,5V & 0,6 & 4,2 & 4,3 \\
\hline
$V_{OH} (V)$ & 4,96V & 4,56 & 4,77 & 5 \\
\hline
$V_{OL} (V)$ & 0,1 & 0,12 & 0,05 & 0,45\\ 
\hline
$NM_{H} (V) $ & 4,06 & 3,66 & 0,27 & 0,4\\ %se pone en VOLTS???????????????????????????
\hline
$NM_{L} (V) $ & 0,4 & 0,48 & 4,15 & 3,85 \\  %se pone en VOLTS???????????????????????????
\hline
$t_{pHL}$ & 87ns & $2,7\mu s $ &  $2,72\mu s$ & 101ns \\
\hline
$t_{pLH}$ & $2,94\mu s$ &  104ns &  73ns &  $2,23\mu s$\\
\hline
$t_{f}$ & 69,5ns & 84ns & 575ns & 770ns\\
\hline
$t_{r}$ & 505ns & 520ns & 83ns & 86 ns\\
\hline
$ I_{Out{Max}}$ &   &   &   &   \\ %AGREGAAAAAAAAAAAAAAAAAAAR
\hline
\end{tabular}

\end{document} 
























