

\documentclass[a4paper,11pt]{report}

%\usepackage[options]{package}
\usepackage{graphicx}
\usepackage{color} 
\usepackage[dvipsnames]{xcolor}
\colorlet{purple}{purple}

\begin{document}

\section{\color{olive}Exercise 1: Design and Implementation of NOT Gates Using Transistors}


\subsubsection{\color{red}High-Level and Low-Level Input Voltages}
The high-level input voltage ($V_{IH}$) es the minimum input voltage that is considered as high, while the low-level input voltage ($V_{IL}$)  is the maximum input voltage that is considered as low.

\subsubsection{\color{red}High-Level and Low-Level Output Voltages}
The high-level output voltage ($V_{OH}$) is the minimum output voltage that the circuit provides as a high, while the low-level output voltage ($V_{OL}$) is the maximum output voltage that the circuit provides as a low.

\subsubsection{\color{red}Noise Margin}
The high noise margin ($NM_{H}$) is the gap between the high-level input voltage and the high-level output voltage, while the low noise margin ($NM_{L}$) is the gap between the low-level output voltage and the low-level input voltage.
$$NM_{H} = V_{OH} - V_{IH}$$
$$NM_{L} = V_{IL} - V_{IH}$$

\subsubsection{\color{red}Propagation Delays}
%ESCRIBIT ACA

\subsubsection{\color{red}Transition Times}
%ESCRIBIT ACA

\subsubsection{\color{red}Maximum Output Current}
%ESCRIBIT ACA

\subsection{\color{purple}Using a BJT NPN 337 Transistor}

\subsection{\color{purple}Using a BJT PNP 327 Transistor}




\end{document} 