<<<<<<< HEAD
\documentclass{article}
\usepackage{graphicx}
\usepackage{siunitx}
\usepackage{csvsimple}
=======
\documentclass[a4paper]{article}
\usepackage{graphicx}
\usepackage{siunitx}
\usepackage{csvsimple}
\usepackage{multirow}
\usepackage{booktabs}
\usepackage[margin=2.5cm]{geometry}
>>>>>>> a165003b37aae4bba9725ea6e802c80dd8b56c59

    \begin{document}
        \section{Implementation of Flip-Flop D and SR Latch with discrete logic gates}
        Using the schematic on Figure \ref{fig:Schem} the logic gates were implemented on a PCB.
        
<<<<<<< HEAD
        \begin{figure}[h]
=======
        \begin{figure}[h!]
>>>>>>> a165003b37aae4bba9725ea6e802c80dd8b56c59
            \begin{center}
                \includegraphics[width=\linewidth]{e6Schem.png}
                \caption{Schematic of the  SR Latch (on the left) and Flip-Flop D (on the right)}
            \end{center}
            \label{fig:Schem}
        \end{figure}

<<<<<<< HEAD
        The resulting circuits were tested and compared to their resulting counterparts as shown in Table \ref{tab:e6res}.
        \begin{table}
            \begin{center}
                
            \end{center}
            \label{tab:e6res}
        \end{table}
            


=======
        The resulting circuits were tested and compared to their resulting counterparts as shown in Table \ref{tab:e6t1}.
        \begin{table}[ht]
            \begin{center}
                \documentclass{article}

\usepackage{booktabs}
\usepackage{siunitx}
\usepackage{pgfplotstable}

\sisetup{
    round-mode      = places,
    round-precision = 2,
}

    \begin{document}
        test text
    \end{document}
            \end{center}
            \label{tab:e6t1}
        \end{table}
        \begin{table}[ht]
            \begin{center}
                \begin{tabular}{|l|l|r|r|r|r|r|r|c|}
    \toprule
    Symbol  &Parameter  &\multicolumn{3}{|c|}{74HC74}&\multicolumn{3}{|c|}{Experimental}&Unit\\
            &           &   MIN&TYP&MAX&MIN&TYP&MAX&\\
    \midrule
    $V_{OH}$&High-level output voltage&3.84&4.3&$-$&  &  &$-$&V\\
    $V_{OL}$&Low-level output voltage &$-$&0.17&0.4&$-$&  &  &V\\
    \bottomrule
\end{tabular}
\caption{Electrical Characteristics comparison}
            \end{center}
            \label{tab:e6t2}
        \end{table}
        
>>>>>>> a165003b37aae4bba9725ea6e802c80dd8b56c59
    \end{document}