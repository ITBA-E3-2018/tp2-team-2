\documentclass[a4paper]{article}
\usepackage{graphicx}
\usepackage{siunitx}
\usepackage{csvsimple}
\usepackage{multirow}
\usepackage{booktabs}
\usepackage{amsmath,amsfonts,mathtools}
\usepackage[margin=2.5cm]{geometry}

    \begin{document}
        \section{Implementation of Flip-Flop D and SR Latch with discrete logic gates}
        Using the schematic on Figure \ref{fig:Schem} the logic gates were implemented on a PCB.
        
        \begin{figure}[h!]
            \begin{center}
                \includegraphics[width=\linewidth]{e6Schem.png}
                \caption{Schematic of the  SR Latch (on the left) and Flip-Flop D (on the right)}
            \end{center}
            \label{fig:Schem}
        \end{figure}

        The resulting circuits were tested and compared to their resulting counterparts as shown in
        Tables \ref{tab:e6t1}, \ref{tab:e6t2}, \ref{tab:e6t3} and \ref{tab:e6t4}.
        \begin{table}[ht]
            \begin{center}
                \documentclass{article}

\usepackage{booktabs}
\usepackage{siunitx}
\usepackage{pgfplotstable}

\sisetup{
    round-mode      = places,
    round-precision = 2,
}

    \begin{document}
        test text
    \end{document}
            \end{center}
            \label{tab:e6t1}
        \end{table}
        \begin{table}[ht]
            \begin{center}
                \begin{tabular}{|l|l|r|r|r|r|r|r|c|}
    \toprule
    Symbol  &Parameter  &\multicolumn{3}{|c|}{74HC74}&\multicolumn{3}{|c|}{Experimental}&Unit\\
            &           &   MIN&TYP&MAX&MIN&TYP&MAX&\\
    \midrule
    $V_{OH}$&High-level output voltage&3.84&4.3&$-$&  &  &$-$&V\\
    $V_{OL}$&Low-level output voltage &$-$&0.17&0.4&$-$&  &  &V\\
    \bottomrule
\end{tabular}
\caption{Electrical Characteristics comparison}
            \end{center}
            \label{tab:e6t2}
        \end{table}
        \begin{table}[ht]
            \begin{center}
                \begin{tabular}{|l|l|l|r|r|r|r|c|}
    \toprule
    Symbol  &\multicolumn{2}{|l|}{Parameter}&\multicolumn{2}{|c|}{74HC74}&\multicolumn{2}{|c|}{Experimental}&Unit\\
            &\multicolumn{2}{c|}{ }&MIN&MAX&MIN&MAX&\\
    \midrule
    $f_{clock}$&Clock frequency&    &$-$&21&$-$&  &MHz\\
    \hline
    \multirow{2}{*}{$t_{w}$}&\multirow{2}{*}{Pulse Duration}& $\overline{PRE}$ or $\overline{CLK}$ low&100&$-$&  &$-$  &\multirow{4}{*}{ns}\\
        &&CLK high or low&100&$-$&  &$-$  &\\
    \multirow{2}{*}{$t_{su}$}&\multirow{2}{*}{Setup time before CLK $\uparrow$}& Data &100&$-$&  &$-$  &\\
    &&$\overline{PRE}$ or $\overline{CLK}$ inactive&100&$-$&  &$-$  &\\
    \bottomrule
\end{tabular}
\caption{Timing Requirements comparison at $V_{CC}$=4.5V}
            \end{center}
            \label{tab:e6t3}
        \end{table}
        \begin{table}[ht]
            \begin{center}
                \begin{tabular}{|l|l|l|r|r|r|r|r|r|c|}
    \toprule
    Symbol  &Input  &Output &\multicolumn{3}{|c|}{74HC74}&\multicolumn{3}{|c|}{Experimental}&Unit\\
            &       &       &       MIN&TYP&MAX          &          MIN&TYP&MAX             &    \\
    \midrule
    $f_{max}$&  &   &25&50&$-$&  &  &$-$&V\\
    \multirow{2}{*}{$t_{pd}$}   &$\overline{PRE}$ or $\overline{CLR}$&Q or $\overline{Q}$&$-$&20&58&$-$&   &   &ns\\
                                &CLK&Q or $\overline{Q}$&$-$&20&44&$-$&   &   &ns\\
    $t_{t}$&    &Q or $\overline{Q}$&$-$&8&19&$-$&   &   &ns\\
    \bottomrule
\end{tabular}
\caption{Switching Characteristics comparison at $V_{CC}$=4.5V}
            \end{center}
            \label{tab:e6t4}
        \end{table}
    \end{document}