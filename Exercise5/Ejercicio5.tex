%%% Preamble
\documentclass[paper=a4,fontsize=11pt]{report}


\begin{document}

%\input{caratula.tex}
%
%\pagenumbering{roman}
%\tableofcontents
%\newpage
%\pagenumbering{arabic}
%
%Test Text

\section{\color{olive}Exercise 5: Compatibility between TTL and CMOS}

Connecting the following circuit in figure \ref{fig:ej5ttl} and \ref{fig:ej5cmos} leaving one of the inputs floating having $Vcc = 5V$. 

\begin{figure}[h!]
         \begin{minipage}{.47\linewidth}
        \centering
        \includegraphics[width=.6\linewidth]{./TTL5.png}
        \caption{\color{cyan}TTL Circuit}
        \label{fig:ej5ttl}
        \end{minipage}
         \begin{minipage}{.5\linewidth}
        \centering
        \includegraphics[width=.5\linewidth]{./CMOS5.png}
        \caption{\color{cyan}CMOS Circuit}
        \label{fig:ej5cmos}
    \end{minipage}
\end{figure}

For the 74LS08, the TTL component, the output port value shown was a logical 1 constantly. On the other hand, the 74HC32, the CMOS component, the results were random at a frequency of 50Hz, having variation from being a quadratic function from 0 to 5V, a 

\begin{figure}[h!]
        \centering
        \includegraphics[scale=0.19]{cmos_cerda2.png}\hspace{1cm}
%        \includegraphics[scale=0.19]{HC-LS-2V.png}\\
        \includegraphics[scale=0.2]{cmos_dads.jpeg}\\
		\vspace{0.2cm}
%		\hspace{0.9cm}
	   \includegraphics[scale=0.19]{cmos_ahifdas.jpeg} 
%        \includegraphics[scale=0.19]{HC-LS-2p3V.png}
        \caption{\color{cyan}74HC02 load to 74LS02}
        \label{fig:ej2exhctols}
    \end{figure}


\end{document}